\NeedsTeXFormat{LaTeX2e}
\documentclass[11pt]{article}
%\usepackage{times}
\usepackage{url}
\usepackage{amsmath}
% \usepackage{mathpazo}
\usepackage{xltxtra}
\usepackage{fancyhdr}
\usepackage{xunicode}
\usepackage{hyperref}
\usepackage{amssymb}
\usepackage{amsthm}
\usepackage{mathpartir}
\usepackage{exercise}
%\usepackage{fontspec}
\usepackage{newunicodechar}
\newtheorem{claim}{Claim}[section]
\newunicodechar{ρ}{\ensuremath{\rho}}
\usepackage[nohead,nofoot,vmargin=1.25in,hmargin=1.25in]{geometry}

\setmainfont[Mapping=tex-text]{Bitstream Iowan Old Style BT}


\newcommand{\deftech}[1]{\textbf{#1}}

\newcommand\mrel{\mathop{\mathbf{r}}}
\newcommand\morel{\mathop{\mathbf{r}'}}
\newcommand\menv{\rho}
\newcommand\mval{v}
\newcommand\mans{a}
\newcommand\mint{i}
\newcommand\moint{j}
\newcommand\mbool{b}
\newcommand\Bool{\mathit{Bool}}
\newcommand\Int{\mathit{Int}}

\newcommand\plug[2]{#1[#2]}

\newcommand\mexp{e}
\newcommand\merr{r}

\newcommand\hole{\Box}
\newcommand\mctx{\mathcal{C}}
\newcommand\mectx{\mathcal{E}}

\newcommand\beval[3]{{#1} \vdash {#2} \Downarrow {#3}}

\newcommand\dom{\mathsf{dom}}

\newcommand{\syntax}[1]{{\tt #1}}

\newcommand\Err{\mathit{Err}}
\newcommand\Plus{\mathit{Plus}}
\newcommand\Mult{\mathit{Mult}}
\newcommand\Succ{\mathit{Succ}}
\newcommand\Pred{\mathit{Pred}}
\newcommand\Eq{\mathit{Eq}}
\newcommand\True{\mathit{True}}
\newcommand\False{\mathit{False}}
\newcommand\If{\mathit{If}}
\newcommand\Div{\mathit{Div}}

\newcommand\reduce{\mathop{\mathbf{a}}}

\newcommand\areducename{\mathbf{a}}
\newcommand\areduce[2]{#1\;\areducename\;#2}

\newcommand\step{\rightarrow_\mathbf{a}}
\newcommand\multistep{\rightarrow^\star_\mathbf{a}}



\newcommand\astdstep{\longmapsto_{\reduce}}
\newcommand\astdmultistep{\longmapsto^\star_{\reduce}}

\newcommand\breducename{\mathbf{b}}
\newcommand\bvreducename{\mathbf{bv}}
\newcommand\bstepname{\rightarrow_{\breducename}}
\newcommand\bmultistepname{\rightarrow^\star_{\breducename}}

\newcommand\bmultistep[3]{#1\vdash #2\;\bmultistepname\;#3}
\newcommand\bstdstepname{\longmapsto_{\breducename}}
\newcommand\bstdstep[3]{#1\vdash #2\;{\longmapsto_{\breducename}}\;#3}
\newcommand\bstdmultistep[3]{#1\vdash #2\;{\longmapsto^\star_{\breducename}}\;#3}

\newcommand\breduce[3]{#1 \vdash {#2}\;\breducename\; {#3}}
\newcommand\bvreduce[3]{#1 \vdash {#2}\;\bvreducename\; {#3}}
\newcommand\bstep[3]{#1 \vdash {#2}\;\rightarrow_{\breducename}\; {#3}}
\newcommand\bclosedstep[2]{{#1}\;\rightarrow_{\breducename}\; {#2}}

\newcommand\laxparstep{\rightrightarrows_\mathbf{a}}
\newcommand\maxparstep{\rightrightarrows'_\mathbf{a}}


\newcommand\Arith{\mathcal{A}}
\newcommand\Barith{\mathcal{B}}
\newcommand\Farith{\mathcal{F}}

\newcommand\Plang{\mathcal{PB}}
\newcommand\PPlang{\mathcal{PB\&J}}

\newcommand\Var{\mathit{Var}}

\newcommand{\mvar}{x}
\newcommand\s[1]{\mathit{#1}}


%% \newcommand\exercise[1]{
%% \vskip 1em
%% \noindent
%% \textbf{Exercise:} {#1}}

%\titlefont{\huge\bfseries}

\begin{document}

\begin{center}
{\bf CMSC631 (Practice) Exam (v2)}\\
{\bf Spring, 2014}
\end{center}


\def\ExerciseName{Problem}

Consider the following syntax of the programming language $\Plang$ (``Peanut Butter''):

\newcommand\Cons{\mathit{Pair}}
\newcommand\Nil{\mathit{Nil}}
\newcommand\IsNil{\mathit{IsNil}}
\newcommand\ProjL{\mathit{Proj_L}}
\newcommand\ProjR{\mathit{Proj_R}}
\newcommand\Cond{\mathit{Cond}}
\newcommand\List{\mathit{List}}

\[
\begin{array}{lrcll}
 \mathit{Exp} 
               & e & ::= & \True\ |\ \False\\
               &   & |   & \Nil\\
               &   & |   & \Cons(\mexp,\mexp)\\
               &   & |   & \ProjL(e)\\
               &   & |   & \ProjR(e)\\
               &   & |   & \Cond(e,e,e)
\end{array}
\]

The Peanut Butter language contains pairs, nil, and booleans.  The behavior
of $\Plang$ programs is a bit quirky; we describe how $\Plang$
programs work informally below:

\begin{itemize}
\item Values in $\Plang$ include booleans, nil, and pairs of values
  (note: this is a recursive definition); pairs are constructed with
  $\Cons$.

\item Pair values are deconstructed with $\ProjL$ and $\ProjR$, which
  project out the left and right component of a pair, respectively.
  So for example, $\ProjL(\Cons(\mexp_1,\mexp_2)) = \mexp_1$.
  Applying $\ProjL$ or $\ProjR$ to non-pair values is an error.

\item $\Cond(\mexp_1,\mexp_2,\mexp_3)$ is a conditional form, which
  selects $\mexp_2$ to evaluate whenever $\mexp_1$ evaluates to a
  \emph{truish} value, and selects $\mexp_2$ to evaluate otherwise.  A
  truish value is any value that is not $\False$.

  So for example, $\Cond(\False, \mexp_1, \mexp_2) = \mexp_2$, but
  $\Cond(\Cons(\False,\False),\mexp_1,\mexp_2) = \mexp_1$.
\end{itemize}


\newpage
\begin{exercise}
Give a formal definition of the set of values in $\Plang$.
\end{exercise}
\vspace{1.2in}

\begin{exercise}
Define a natural semantics for $\Plang$. Show the derivation for
exaluating the program:
\[
\ProjL(\Cond(\Cons(\True,\False),\Cons(\False,\Nil),\Cons(\True,\Nil)))
\]
%
(Your semantics should only specify the ``good'' behavior of programs
and doesn't need to bother with erroneous programs.)
\end{exercise}

\newpage
It turns out that even though $\Plang$ only has pairs, nil, and
booleans for values, $\Plang$ programmers tend to think in terms of
``lists''.  Lists are either empty or consist of an element paired
together with another list.  An empty list is represented by $\Nil$.  So
for example, the list of three $\True$ values could be represented:
\[
\Cons(\True,\Cons(\True,\Cons(\True,\Nil)))\text.
\]
%
Moreover, $\Plang$ programmers think in terms of \emph{homogeneous}
lists, i.e. lists of the same kinds of elements.  So for example, a
$\Plang$ programmer thinks in terms of ``a list of booleans'' or ``a
list of lists of booleans,'' etc.

With that in mind we can formalize a notion of types for $\Plang$:
\[
\begin{array}{lrcll}
 \mathit{Type} 
               & t & ::= & \Bool\\
               &   &  |\ & \List(t)
\end{array}
\]

\vspace{.5in}

\begin{exercise}
Define a type judgement relation for $\Plang$ programs.  Your type
system should accept the program given in problem 2 as having type
$\Bool$.  Give the type derivation for the program in problem 2.  Give
an example of a program that is ill-typed.
\end{exercise}


\newpage
After years of use, the $\Plang$ language was replaced by it's
successor $\PPlang$, which added the following features to $\Plang$:

\begin{itemize}
\item Using the $\mathit{J}(\mexp)$ operator, programs could jump to end of
  evaluation, making the value of $\mexp$ the final result of the computation.

\item Programs no longer consisted of single expressions $\mexp$, but
  instead consist of any number function definitions followed by an
  expression that can make use of those definitions.  Functions take a
  single argument and may be (mutually) recursive.  Functions are
  \emph{not} values in $\PPlang$.

\item Projection operations were replaced by a pattern maching
  construct: $\mathit{Let}(x,y,e_1,e_2)$ which evaluates $e_1$ to a
  pair then binds $x$ to the left component and $y$ to the right,
  within the scope of $e_2$.
\end{itemize}

\vspace{.5in}

\begin{exercise}
Give a formal definition of the syntax of $\PPlang$ programs.
\end{exercise}

\newpage

\begin{exercise}
Define a small step reduction semantics for $\PPlang$.
\end{exercise}

\end{document}

